\documentclass{article}

\usepackage[utf8]{inputenc}
\usepackage[T1]{fontenc}
\usepackage{graphicx}
\usepackage{float}

\title{Translation English to French}
\author{Mohamad Lakkis}
\date{\today}

\begin{document}

\maketitle

\begin{abstract}
  This document explains the model used to translate English text to French text, presenting the model's results and discussing its limitations. This study provides insights for understanding sequence-based NLP tasks, particularly in the context of translation using recurrent neural networks.
\end{abstract}
  
\section{Introduction}
\ \ \ This translation model was trained in two different settings: one using unidirectional and the other using bidirectional LSTM networks to convert English text to French. Although this basic LSTM architecture produces reasonably not bad results, we explore potential improvements using an attention mechanism, which we leave for future work. For now, we focus on LSTM networks with multiple layers, as will be discussed later.\\

The goal of these simulations is to enable the model to grasp the semantic meaning of phrases, rather than performing a simple word-by-word translation. To accomplish this, we selected an encoder-decoder architecture: the encoder captures the semantic structure and order of the words, while the decoder generates the French translation. In future work, we plan to enhance this model by introducing an attention mechanism and expanding the dataset.\\

\section{Methodology}
In this section, we describe the model's architecture and the training process. The model consists of an encoder and a decoder, both of which are LSTM networks. The encoder processes the input sequence, while the decoder generates the output sequence. The model is trained using the Adam optimizer and the categorical cross-entropy loss function.\\
\subsection{Data Preprocessing}
As a first step, we need an english and french vocabulary, so we tokenize the input sentences (in this case we used a tokenizer for each word, from NLTK library) and then formed the english vocabulary, from these disticnt tokens. After that we start indexing each token in the input sequences. At this step each token is represnted by an integer, so consequently each sentence is represented by a sequence of integers.\\
The same prcoess is done for getting the french vocabulary and indexing the output sequences.\\
Additionally, we will also get the inverse maping of the french and english vocabs, to be able to convert the output sequences from the model to the corresponding words.\\
Now in each of the two vocabs (english and french) we will add to them 4 speical tokens: 
\begin{itemize}
    \item <PAD> token: to pad the sequences to have the same length.
    \item <UNK> token: to represent the unknown words.
    \item <EOS> token: to represent the end of the sentence.
    \item <SOS> token: to represent the start of the sentence.
\end{itemize}
Now we for the output sequences we will add the <SOS> token at the start of each sequence and the <EOS> token at the end of each sequence.\\ Note: we will not add these tokens to the input sequence.\\
Furthermore, we will pad the input sequences to have the same length (as maximum lenth of the input sequences) so any input sequence that is shorter than the maximum length will be padded with the <PAD> token. And any input sequence that is longer than the maximum length will be truncated.\\ Similar steps will be done for the output sequences.\\ 
Not that for the input sequence we will not be adding <EOS> and <SOS> tokens. We will use another method to process only the unpadded part of the sequence, and for that we will need the src\_lenth, which will denote the actual lenth of the each sequence before padding.\\
After this step our inputs and expected outputs are ready to be fed to the model. \\ 
\textbf{So to wrap up: Our inputs for the models will be the padded input sequences and the src\_length. As for its output (during training) will be probablity distribution over the french vocab, as for the output during inference it is just the class (the french token, that is most probable).}\\ \\ 
\begin{figure}[H]
    \centering
    \includegraphics[width=0.9\textwidth]{Images/DataPreprocessing.png}
    \caption{Data Preprocessing}
\end{figure}
\textbf{Important Note: Both Inputs and Outputs are common for both models.}
\subsection{Model 1 Architecture: One-Directional LSTMs }
Now that we understand the data preprocessing, we can move to the model architecture.\\
I think it is better now to look at the general hierachy of the model through the following figure.\\
\begin{figure}[H]
    \centering
    \includegraphics[width=0.9\textwidth]{Images/2.png}
    \caption{General Overview of the Model}
\end{figure}
\section{Importance of training and ways to prevent overfitting}
\begin{figure}[H]
    \centering
    \includegraphics[width=0.8\textwidth]{Images/model_1_1.png}
    \caption{Correct Output}
\end{figure}
\begin{figure}[H]
    \centering
    \includegraphics[width=0.8\textwidth]{Images/model_1_2.png}
    \caption{Overfitting and Association}
\end{figure}
From these two figures we can see how the model learned to associate the word truck always with color jaune, so even when the input is just truck the model will still output the "yellow truck" in french. \\ This could be solved by adding more data, of by using additional layers with dropout, or also by using the attention mechanism which we will explore next. 
\section{Importance of Attention Mechanism}
\begin{figure}[H]
    \centering
    \includegraphics[width=0.9\textwidth]{Images/importance_attention.png}
    \caption{Attention Mechanism}
\end{figure}
this example clearly illustrates the importance of incorporating an attention mechanism in a Seq2Seq model. Without attention, the model processes the sentence sequentially, which can lead it to focus only on certain parts, especially the beginning or end, while losing context in the middle or when shifts in meaning occur (e.g., handling both "I dislike bananas" and "she likes grapefruit").\\ 
In this example, the model incorrectly translates "dislike" to "like" (in French, "je n'aime pas" would be used for "dislike" instead of "j'aime") and ignores the contrast between "I" and "she," leading to an incorrect translation.\\
The attention mechanism allows the model to dynamically focus on relevant parts of the input sequence as it generates each word in the output sequence. This enables the model to capture the context and meaning of the entire sentence, leading to more accurate translations.\\
The topic of attention will be explore in future work, as it is a crucial component for improving the model's performance.
\section{Conclusion}
Your conclusion here.

\bibliographystyle{plain}
\bibliography{references}

\end{document}